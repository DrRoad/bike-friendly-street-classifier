\documentclass[10pt,twocolumn,letterpaper]{article}

\usepackage{cvpr}
\usepackage{times}
\usepackage{epsfig}
\usepackage{graphicx}
\usepackage{amsmath}
\usepackage{amssymb}

% Include other packages here, before hyperref.

% If you comment hyperref and then uncomment it, you should delete
% egpaper.aux before re-running latex.  (Or just hit 'q' on the first latex
% run, let it finish, and you should be clear).
\usepackage[breaklinks=true,bookmarks=false]{hyperref}

\cvprfinalcopy % *** Uncomment this line for the final submission

\def\cvprPaperID{****} % *** Enter the CVPR Paper ID here
\def\httilde{\mbox{\tt\raisebox{-.5ex}{\symbol{126}}}}

% Pages are numbered in submission mode, and unnumbered in camera-ready
%\ifcvprfinal\pagestyle{empty}\fi
\begin{document}

%%%%%%%%% TITLE
\title{Title}

\author{First Author\\
Institution1\\
Institution1 address\\
{\tt\small firstauthor@i1.org}
% For a paper whose authors are all at the same institution,
% omit the following lines up until the closing ``}''.
% Additional authors and addresses can be added with ``\and'',
% just like the second author.
% To save space, use either the email address or home page, not both
\and
Second Author\\
Institution2\\
First line of institution2 address\\
{\tt\small secondauthor@i2.org}
}

\maketitle
%\thispagestyle{empty}

%%%%%%%%% ABSTRACT
\begin{abstract}
   It should not be more than 300 words.
\end{abstract}

%%%%%%%%% BODY TEXT
\section{Introduction}
This section introduces your problem, and the overall plan for approaching your problem. 

\section{Background/Related Work}
This section discusses relevant literature for your project.  

\section{Approach}
This section details the framework of your project. Be specific, which means you might want to include equations, figures, plots, etc

\section{Experiment}
This section begins with what kind of experiments you're doing, what kind of dataset(s) you're using, and what is the way you measure or evaluate your results. It then shows in details the results of your experiments. By details, we mean both quantitative evaluations (show numbers, figures, tables, etc) as well as qualitative results (show images, example results, etc).

\section{Conclusion}
What have you learned? 

Suggest future ideas.\\

List and number all bibliographical references at the end of your paper. When referenced in the text,
enclose the citation number in square brackets, for
example. \\


\cite{slavkovikj2014image}
\cite{patterson2014sun}
\cite{rundle2011using}
\cite{zhou2017places}
\cite{DBLP:journals/corr/ZhouKLOT14}
\cite{DBLP:journals/corr/Wang15l}
\cite{DBLP:journals/corr/HeZRS15}
\cite{DBLP:journals/corr/ZagoruykoK16}
\cite{DBLP:journals/corr/ZhouKLTO16}
\cite{koehrsen2018blog}

{\small
\bibliographystyle{ieee}
\bibliography{cs682_final_report.bib}
}

\end{document}
